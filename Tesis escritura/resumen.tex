\begin{resumen}%
    A pesar de ser una de las especies más comercializadas en el mercado ilegal de mascotas de Argentina, se conoce muy poco sobre la tortuga 
    terrestre \textit{Chelonoidis chilensis} en su hábitat natural. Debido 
    además a la creciente fragmentación de su hábitat producida principalmente 
    por la reciente introducción de ganado, esta tortuga está catalogada como 
    especie en estado 
    vulnerable. Por estos motivos resulta muy importante el 
    estudio de sus refugios, su área de movimiento y las relaciones entre 
    las tortugas dentro la comunidad. La población de estudio se encuentra 
    en el límite sur de su distribución geográfica, en las cercanías de
    San Antonio Oeste, Patagonia, Argentina. Por ser reptiles se 
    consideran 
    solitarios, aunque se sabe muy poco sobre su red de interacción social. 
    En este trabajo, se estudió el movimiento de 27 individuos 
    de \textit{Chelonoidis chilensis} usando dos técnicas de monitoreo: 
    una unidad de navegación  autónoma con GPS y un \textit{datalogger} comercial. 
    Se implementó un método de filtrado de trayectorias y se construyó una 
    grilla de zonas de interés para las tortugas, utilizando las 
    trayectorias filtradas e interpoladas. Se implementaron dos criterios 
    para identificar los refugios nocturnos de las tortugas. Sobre éstos 
    se calculó la distancia media entre refugios y su centro de masa, 
    tanto para machos como para hembras y no se 
    encontraron 
    diferencias significativas en el área abarcada por los refugios para 
    ambos sexos. Se armaron redes bipartitas de nodos 
    tortuga y refugio, y se compararon las proyecciones en nodos tortuga, 
    con las redes de interacción diurnas entre pares de tortugas. Se utilizó 
    la proyección en nodos tortuga de 
    la red bipartita como predictor de enlaces en la red de encuentros diurnos
    y se encontró que las predicciones no son estadísticamente significativas. 
    Se calcularon métricas sobre la topología de la red proyectada y no se 
    encontraron diferencias respecto del uso aleatorio de refugios. Finalmente
    , se descubrió la existencia de refugios preferidos y que 
    la tortuga pasa noches consecutivas en refugios geográficamente cercanos.
\end{resumen}

\begin{abstract}%
    Although it is one of the most comercialized species in the Argentinean 
    iligal
    pet market, very little is known about the \textit{Chelonoidis chilensis} 
    tortoise in the wild. This fact, together with the increasing habitat fragmentation 
    produced by cattle, that is being recently introduced into the area, lead to the 
    classification of its conservation status as vulnerable. It is therefore escencial to 
    to learn about their burrows, their movement area and the relationship 
    between themselves and within their community. The studied population lives at the
    southernmost distribution of the species, near to San Antonio Oeste city in Patagonia, Argentina. 
    As they are reptiles, they are considered mostly solitary, although very little is 
    known about their social interaction network. In this work, the 
    movement of 27 \textit{Chelonoidis chilensis} tortoises were studied using two monitoring 
    techniques: an autonomous GPS navigation unit and a commercial \textit{datalogger}. A 
    trajectory filtering method was implemented and a grid was built, showing the 
    tortoises's areas of interest. Two criteria were used to identify tortoise nocturnal 
    burrows. The average distance between burrows to the center of mass 
    (of these burrows) was calculated for males and females and no significant differences
    were found. Bipartite networks of tortoise nodes and burrows were built and 
    the projection in tortoise nodes was compared with the diurnal interaction network. The 
    projection 
    in tortoise nodes of the bipartite network was used as a predictor of links 
    in the encounter diurnal network and it was found that the predictions are not 
    statistically significant. Metrics on the topology of the projected network were 
    calculated and no differences were found with respect to the random use of burrows. 
    Finally, the existence of preferred burrows was found and also that tortoises spend
    consecutive nights in nearby burrows.
\end{abstract}


%%% Local Variables: 
%%% mode: latex
%%% TeX-master: "template"
%%% End: 
