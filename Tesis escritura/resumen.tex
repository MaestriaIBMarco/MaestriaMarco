\begin{resumen}%
    Sobre la tortuga \textit{Chelonoidis chilensis} se sabe muy poco, es una especie en estado vulnerable  y actualmente, en la zona de estudio, se esta introduciendo ganado; haciendo muy importante el estudio de  sus refugios, su area de movimiento y las relaciones entre tortugas dentro la comunidad. La población de estudio está en la distribución más al sur. Además, por ser reptil se consideran solitarios y se sabe muy poco sobre su red de interacción social. En este trabajo, se estudió el movimiento de 27 individuos de \textit{Chelonoidis chilensis} usando dos técnicas de monitoreo: una unidad de navegación  autónoma con GPS y un datalogger comercial. Se implementó un método de filtrado de trayectorias y se construyó una grilla de zonas de interés para las tortugas, utilizando las trayectorias filtradas e interpoladas. Se implementaron dos criterios para identificar los refugios nocturnos de las tortugas. Sobre estos refugios se calculó la distancia media entre refugios al centro de masa (de estos refugios) para machos y hembras y no se encontraron diferencias significativas. Se armaron redes bipartitas de nodos tortugas y refugios y se compararon las proyecciones en nodos tortugas con redes de interacción, armadas a través de los encuentros medidos entre pares de tortugas. Se utilizó la proyección en nodos tortugas de la red bipartita como predictor de enlaces en la red de encuentros y se encontró que las predicciones no son estadísticamente significativas. Se calcularon métricas sobre la topología de la red proyectada y no se encontraron diferencias respecto al uso aleatorio de refugios. Se observaron la existencia de refugios preferidos y se concluyó que la tortuga pasa la noche en refugios cercanos respecto a otros refugios medidos en la zona de medición.
\end{resumen}

\begin{abstract}%
    Very little is known about the Chelonoidis chilensis tortoise, a specie in a vulnerable situation.  Currently, cattle is being introduced in the area where it is studied, making it very important to learn about their burrows, their movement area and the relationship between themselves within their community. The study population is in the southernmost distribution. As they are reptiles, they are considered solitary and very little is known about their social interaction network. In this work, the movement of 27 Chelonoidis chilensis  were studied using two monitoring techniques: an autonomous GPS navigation unit and a commercial datalogger. A trajectory filtering method was implemented and a grid was built showing the tortoises areas of interest once the trajectories were filtered. Two criteria were used to identify the tortoises night burrows. The average distance between burrows to the center of mass (of these burrows) was calculated for males and females and no significant differences were found. Bipartite networks of tortoise nodes and burrows were built and the projections in tortoise nodes were compared with interaction networks, built through the measured encounters between pairs of tortoises. The projection in tortoise nodes of the bipartite network was used as a predictor of links in the encounter network and it was found that the predictions are not statistically significant. Metrics on the topology of the projected network were calculated and no differences were found regarding the random use of burrows. The existence of preferred burrows was observed and it was concluded that the tortoise spends the night in nearby burrows compared to other burrows measured in the measurement area.
\end{abstract}


%%% Local Variables: 
%%% mode: latex
%%% TeX-master: "template"
%%% End: 
