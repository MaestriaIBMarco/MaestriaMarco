\chapter{Conclusiones y trabajo a futuro}
 
Se diseñó un algoritmo de reconstrucción de trayectorias, el mismo puede ser usado en cualquier especie indicando la velocidad máxima de la misma. Se encontró que la velocidad máxima de las tortugas es de 14 m/min. Fueron identificadas las zonas más visitadas por la población de tortugas estudiadas (Fig.~\ref{fig:grillaInt}), lo que podría ser útil para diseñar políticas de conservación. Los mapas de recurrencias podrán ser actualizados con nuevas mediciones y al devolver mapas en un formato html interactivo son de fácil uso para un trabajo interdisciplinario \LGM{, en particular, durante las campañas?}.
 
Se obtuvieron redes  de interacción de individuos (Figs. \ref{fig:redInteraccion20mincampanas} y \ref{fig:redInteraccion20igotu}) y la cantidad de encuentros promedio para machos y hembras en los distintos meses de medición (Figs.~\ref{fig:encuentros_hora_medida_tortugometro} y \ref{fig:encuentros_hora_medida_igotu}). Se observa un pico para ambos sexos en noviembre-diciembre para los datos del tortugometro, donde el 85\% de los encuentros registrados fueron hembra-macho, éste puede estar asociado a la búsqueda de pareja en la época de apareamiento. En el mes de enero las hembras presentan una mayor cantidad de encuentros promedio por hora medida que los machos (para datos de i-gotU y tortugometro), \LGM{esta obervación} se estima \LGM{que puede} estar relacionada a la búsqueda de un lugar para sus huevos. De todas maneras, solo se contaron con 8 tortugómetros y 6 i-gotU  para cada día midiendo en simultáneo y es probable que parte de los encuentros no hayan sido registrados. En un futuro se analizará la red de interacciones aumentando el número de pares de individuos monitoreados.
 
 
Se definieron dos criterios para determinar el refugio donde pasó la noche la tortuga, uno para cada metodología de medición. Reconociendo las limitaciones de estos criterios, se decidió para las próximas campañas de medición aumentar las franja horaria de medición para los datos de i-gotU, de esta manera garantizar la ubicación de la tortuga en el refugio nocturno. Para el caso del tortugómetro se decidió añadir una etiqueta a los días donde se recupera el dispositivo en un refugio \LGM{no queda claro qué información tendría la etiqueta}.
 
Con los refugios ya identificados, se definió una métrica para determinar la distribución espacial de los refugios en tortugas machos y hembras. Se observó que no presentan diferencias significativas entre machos y hembras, de todas maneras, el criterio utilizado para determinar un refugio para datos del tortugometro filtra muchos días de medición donde se le quito \LGM{acento} el tortugometro previo a las 20 horas. Por otro lado, los datos de i-gotU (donde tenemos una gran cantidad de refugios registrados) están tomados en los meses de enero-mayo, donde se espera una menor actividad de las tortugas machos (pasada época de apareamiento)\cite{Erika}.
 
Se identificaron en los refugios monitoreados por los i-gotU, donde tenemos días consecutivos de medición, la existencia de refugios preferidos (Fig. \ref{fig:refugios_preferidos}) donde la tortuga pasa la mayoría de las noches. En varias tortugas \LGM{se podría poner más o menos cuántas, aunque sea aproximado?}, se encontró que realizan caminatas desde el refugio preferido hasta otro refugio nocturno y luego vuelven al refugio preferido. Un ejemplo de este último es la tortuga T54, en la red que manifiesta los caminos tomados (Fig. \ref{fig:ruta_refus_T54}), se observa que la tortuga T54 \LGM{no hace falta repetir el nombre, ya se dijo en la misma oración al principio} toma varias caminatas a otros refugios nocturnos y luego vuelve a alguno de sus dos refugios preferidos. Caracterizar estos refugios preferidos puede ser útil para diseñar políticas de conservación \LGM{y puede además contribuir a la caracterización del movimiento, tanto para modelos matemáticos como simulaciones numéricas}.
 
Partiendo de los refugios asociados a cada tortuga se armaron redes bipartitas de refugios y tortugas (Figs. \ref{fig:red_bipartita_refus_campanas} y \ref{fig:red_bipartita_refus_igotu}). Utilizando las proyecciones de los grafos bipartitos como predictor de enlaces en la red de encuentros, se encontró que para los datos de los tortugometros las métricas recall, precisión y accuracy no son mayores de lo esperado por uso de refugios aleatorio (tabla \ref{tab:metricas_comparacion_redes_aleatorias}). Para los datos de i-gotU, se observa que las métricas son mayores a las esperadas por uso de refugios aleatorios, pero se están monitoreando solo 6 tortugas y únicamente se encontraron 2 refugios compartidos entre las tortugas. Por poca cantidad de datos, no se puede \LGM{aún} afirmar si las proyecciones en la red bipartita nos dan una predicción de la red de encuentros. En un futuro se analizará este razonamiento en \LGM{con} una mayor cantidad de tortugas donde se disponga \LGM{la} misma cantidad de datos de refugios que \LGM{de} días de medición (condición que \LGM{actualmente} no se cumple para los datos del tortugometro).
 
 
 
 
 
 
 
 
 
 
%A partir de los resultados de este estudio, se decidió realizar en la próxima campaña un mayor esfuerzo de seguimiento de los individuos que participaron de los encuentros  detectados. De esta forma se busca entender si la no repetición de encuentros entre tortugas se debe a la poca cantidad de muestras o a una característica de la especie. Este resultado sería muy novedoso dado que está relacionado con la capacidad de memoria de las tortugas, un aspecto muy poco estudiado hasta el momento.
 

